\chapter{نیازمندی‌های اتصالات خارجی}
\clearpage
\noindent {
این قسمت حاوی اطلاعات مربوط به هر گونه ورودی و خروجی مربوط به فعالیت کاربران(رابط کاربری) در دایکر است.
ضمنا جزییات اتصالات نرم‌افزاری و سخت‌افزاری دیگر سامانه‌ها به دایکر نیز
شرح داده می‌شود.
}

\section {رابط کاربری}
\noindent {
رابط کاربری دایکر یک
\lr{Web-based Dashboard}
است که کاربر پس از لاگین به آن می‌تواند به کمک این
\lr{Dashboard}
به تحلیل داده‌ها بپردازد.
}

\begin{figure}[!ht]
  \centering 
  \includegraphics[width=0.8\textwidth]{figures/UI}
  \caption[رابط کاربری دایکر]
  {قالب رابط کاربری دایکر برای استفاده‌ی کاربران عادی}
\end{figure}

همین طور که در شکل بالا می‌بینید رابط کاربری دایکر در زمان کاربر عادی از سه جز ساخته می‌شود.
نوار بالایی که در بالا‌ی صفحه دیده‌ می‌شود و حاوی لوگوی دایکر است. فضای گزارش که نمودار‌ها و اطلاعاتی که باید
از کاربر گرفته شود و یا به وی نشان داده شود درون آن خواهد آمد.
سرانجام در قسمت چپ فضای کاربری فضایی برای کنترل تعامل کاربر با دایکر به کمک منو‌های مختلف
دیده می‌شود.

\begin{figure}[!ht]
  \centering 
  \includegraphics[width=0.9\textwidth]{figures/ui5}
  \caption[تصویر واسط کاربری ۱]
  {نمونه تصویر رابط کاربری ۱}
\end{figure}

\begin{figure}[!ht]
  \centering 
  \includegraphics[width=0.9\textwidth]{figures/ui9}
  \caption[تصویر واسط کاربری ۲]
  {نمونه تصویر رابط کاربری ۲}
\end{figure}

\begin{figure}[!ht]
  \centering 
  \includegraphics[width=0.9\textwidth]{figures/ui7}
  \caption[تصویر واسط کاربری ۳]
  {نمونه تصویر رابط کاربری ۳}
\end{figure}

\begin{figure}[!ht]
  \centering 
  \includegraphics[width=0.9\textwidth]{figures/ui11}
  \caption[تصویر واسط کاربری ۴]
  {نمونه تصویر رابط کاربری ۴}
\end{figure}

\clearpage
\section {رابط سخت‌افزاری}
\noindent {
دایکر برای اجرای خود به سخت افزار خاصی وابسته نیست.
در صورت وجود وابستگی به سخت افزار خاص،‌ این وابستگی به
سیستم‌ عامل و
\lr{Run-time System}
ای‌ که دایکر در آن در حال اجراست برمی‌گردد.
}

\section {رابط نرم‌افزاری}
\noindent {
دایکر برای کار نیازمند به اتصال به دیتابیس اطلاعات خرید مشتریان است.
دایکر پس از اتصال به دیتابیس اطلاعات مورد نیاز خودش را درون
\lr{Caching Database}
خود
\lr{Import}
می‌کند.
بدین صورت عملیات روی دیتابیس اطلاعات خرید مشتریان به اعمال پرس‌ وجو‌های
\LTRfootnote{\lr{Queries}}
مناسب دریافت رکود‌ها محدود خواهد بود.
}

