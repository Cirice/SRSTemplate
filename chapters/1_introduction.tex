\chapter {مقدمه}
\clearpage

\noindent {
این قسمت شامل معرفی قسمت‌های مختلف این نوشته به همراه تعریف چشم‌انداز و محدوده‌ی کلی بحث شده در این سند است.
ضمنا در این قسمت هدف وجود این سند به همراه تعاریف عبارات مطرح شده در این سند به
خواننده عرضه خواهد شد.
}

\section{هدف این سند}
\noindent {
در این قسمت باید هدف نوشتن این سند را توضیح دهید.
}

\section{محدوده‌ی کار سامانه}
\noindent {
در این قسمت محدوده(\lr{Scope}) کار سیستم تحت توسعه را شرح دهید.
منظور از محدوده، خصوصیات کار یا سرویسیست که نرم‌افزار برآورده می‌کند.
}

\section {تعاریف مهم}
\noindent {
در این قسمت تعریف هر عبارتی که برای خواننده بروز ابهام کند آورده می‌شود.
}

\begin{description}

\item \textbf{کاربر}:
فردی که از نرم‌افزار استفاده می‌کند.

\item \textbf{\lr{System Image}}:
تصویر لحظه‌ای از
\lr{State}های 
یک سیستم.

\item \textbf{\lr{Work-space}}:
فضایی از رم که به ذخیره سازی متغیر‌ها و
\lr{Object}های
یک برنامه در حال اجر درون رم اختصاص داده شده است.

\item \textbf{\lr{Run-time System}}:
کلیه چیز‌هایی(اعم از کتابخانه‌ها، مدل پردازش و فایل‌ها) که برای اجرای یک
برنامه نیاز هستند.
\lr{Run-time System}
پیاده‌سازی مدل لازم برای اجرا برنامه است.

\end{description}

\section {ارجاع به منابع}
\noindent {
در این قسمت ارجاع‌ها به منابع اطلاعاتی که از اطلاعاتشان در این نوسته استفاده شده
آورده می‌شوند.
}

\pagestyle{empty}
{
\onehalfspacing
\bibliographystyle{ieeetr-fa}%{acm-fa}
\bibliography{bibliography/bibliography}
\nocite{*}
}

\section {ساختار مابقی این سند}
\noindent {
این بخش توضیح می‌دهد که بقیه این نوشته چه ساختاری دارد و در ادامه هر فصل حاوی چه
اطلاعاتیست.
}