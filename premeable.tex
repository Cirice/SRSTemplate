\usepackage{amsthm,amssymb,amsmath}
\usepackage[top=40mm, bottom=40mm, left=25mm, right=35mm]{geometry}
\usepackage{graphicx}
\usepackage{framed} 
\usepackage{lastpage}
\usepackage[pagebackref=false,colorlinks=false,linkcolor=black,citecolor=blue,unicode=true,bookmarks=true]{hyperref}
\usepackage{fancyhdr}
\usepackage{setspace}
\usepackage{algorithm}
\usepackage{algorithmic}
\usepackage{subfigure}
\usepackage{caption}
\usepackage[subfigure]{tocloft}
\usepackage[nottoc]{tocbibind}
\usepackage{makeidx}
\makeindex

\usepackage[export]{adjustbox}
\usepackage{array}
\usepackage{tabularx}
\usepackage{pifont,mdframed}
\usepackage{colortbl}
\usepackage{longtable}
\usepackage{etoc}
\usepackage{minted}
\usepackage[listings,skins,breakable]{tcolorbox}
\usepackage{xcolor}
\usepackage{mdframed}
\usepackage{xepersian}
\setmainfont[Ligatures=TeX, Path=fonts/, Scale=1.2]{XB Niloofar.ttf}
\settextfont[Scale=1.2, Path=fonts/, BoldFont=B Nazanin Bold.ttf]{B Nazanin.ttf}
\setlatintextfont[Scale=0.8, Path=fonts/, BoldFont=timesbd.ttf]{times.ttf}
\setdigitfont[Scale=0.8, Path=fonts/]{XB Zar.ttf}
\defpersianfont\titlefont[Scale=0.8, Path=fonts/]{XB Titre.ttf}
\SepMark{-}

\theoremstyle{definition}
\newtheorem{definition}{تعریف}[section]
\newtheorem{theorem}[definition]{قضیه}
\newtheorem{lemma}[definition]{لم}
\newtheorem{proposition}[definition]{گزاره}
\newtheorem{corollary}[definition]{نتیجه}
\newtheorem{remark}[definition]{ملاحظه}
\theoremstyle{definition}
\newtheorem{example}[definition]{مثال}
\numberwithin{algorithm}{chapter}
\def\listalgorithmname{فهرست الگوریتم‌ها}
\def\listfigurename{فهرست تصاویر}
\def\listtablename{فهرست جداول}

\doublespacing
\newlength\mylenprt
\newlength\mylenchp
\newlength\mylenapp

\renewcommand\cftpartpresnum{\partname~}
\renewcommand\cftchappresnum{\chaptername~}
\renewcommand\cftchapaftersnum{:}

\settowidth\mylenprt{\cftpartfont\cftpartpresnum\cftpartaftersnum}
\settowidth\mylenchp{\cftchapfont\cftchappresnum\cftchapaftersnum}
\settowidth\mylenapp{\cftchapfont\appendixname~\cftchapaftersnum}
\addtolength\mylenprt{\cftpartnumwidth}
\addtolength\mylenchp{\cftchapnumwidth}
\addtolength\mylenapp{\cftchapnumwidth}

\setlength\cftpartnumwidth{\mylenprt}
\setlength\cftchapnumwidth{\mylenchp}	

\makeatletter
{\def\thebibliography#1{\chapter*{\refname\@mkboth
   {\uppercase{\refname}}{\uppercase{\refname}}}\list
   {[\arabic{enumi}]}{\settowidth\labelwidth{[#1]}
   \rightmargin\labelwidth
   \advance\rightmargin\labelsep
   \advance\rightmargin\bibindent
   \itemindent -\bibindent
   \listparindent \itemindent
   \parsep \z@
   \usecounter{enumi}}
   \def\newblock{}
   \sloppy
   \sfcode`\.=1000\relax}}
\makeatother

\usepackage{grfext}
\graphicspath{{figures/}}
\usepackage{caption}

%% new colours
\definecolor{aox}{rgb}{0.0, 0.5, 0.0}
\definecolor{bred}{rgb}{0.8, 0.0, 0.0}
\definecolor{darkolivegreen}{rgb}{0.33, 0.42, 0.18}

\newcommand{\lbb}{\textbf{\{}}
\newcommand{\rbb}{\textbf{\}}}
\newcommand{\gcm}[1]{\textbf{\textcolor{aox}{#1}}}
\newcommand{\egcm}[1]{\textbf{\textcolor{bred}{#1}}}

%% toc depth
\setcounter{tocdepth}{3}
\setcounter{secnumdepth}{3}

%% table coluring
\newcommand{\mc}[2]{\multicolumn{#1}{c}{#2}}
\definecolor{Gray}{gray}{0.85}
\definecolor{LightCyan}{rgb}{0.8,1,1}

\newcolumntype{a}{>{\columncolor{Gray}}c}
\newcolumntype{b}{>{\columncolor{white}}c}

%% waning box
\definecolor{warningbackground}{RGB}{252,226,158}
\newcommand{\alertwarningbox}[1]{
    \begin{centering}
    \begin{tabularx}{\linewidth}{
        >{\columncolor{warningbackground}}c
        >{\columncolor{warningbackground}}X}
        \raisebox{\dimexpr2\baselineskip-\height}
        {\includegraphics[scale=1.0]{figures/warning.jpg}}&
        \raisebox{\tabcolsep}{\strut}#1\raisebox{-\tabcolsep}{\strut}
    \end{tabularx}
    \end{centering}
}

\makeatletter
\def\set@docident{%
  \begingroup
    % yes, a big \ifcase would be more straightforward;
    % no, that wouldn't be any fun at all
    % (also it ends up looking messy and even tricksier).
    \def\@step##1##2\@nil{\advance\@tempcnta##1 \def\@tempa{##2}}
    \def\@tempa{{31}{28}{31}{30}{31}{30}{31}{31}{30}{31}{30}{31}}%
    \@tempcnta=\day             % day of month
    \@tempcntb=\month           % month of year (unit-offset)
    \loop
      \advance\@tempcntb-1
      \ifnum \@tempcntb>0
        \expandafter\@step\@tempa\@nil
    \repeat
    \@tempcntb=\year            % yes, do calculate leap years
    \divide\@tempcntb 4 \multiply\@tempcntb 4
    \ifnum\@tempcntb=\year
      \ifnum\month>2            % but let's not worry about century years (slack...)
        \advance\@tempcnta 1
      \fi
    \fi
    \xdef\docident{%
      \the\@tempcnta            % day-of-year
      -\the\time}               % minutes since midnight
  \endgroup
}
\set@docident
\makeatother

\usepackage[some]{background}
\definecolor{titlepagecolor1}{cmyk}{0.3,0.6,0,.60}
\definecolor{titlepagecolor2}{cmyk}{0.3,0.6,0,.60}

\DeclareFixedFont{\bigsf}{T1}{phv}{b}{n}{1.5cm}

\backgroundsetup{
scale=1,
angle=0,
opacity=1,
contents={\begin{tikzpicture}[remember picture,overlay]
 \path [fill=titlepagecolor1] (-0.5\paperwidth,5) rectangle (0.5\paperwidth,10);
 \path [fill=titlepagecolor2] (-0.5\paperwidth,5) rectangle (0.5\paperwidth,10); 
\end{tikzpicture}}
}